\documentclass[compress,mathserif]{beamer}
\usetheme{sthlm}

%-=-=-=-=-=-=-=-=-=-=-=-=-=-=-=-=-=-=-=-=-=-=-=-=
%        LOADING BEAMER PACKAGES
%-=-=-=-=-=-=-=-=-=-=-=-=-=-=-=-=-=-=-=-=-=-=-=-=

\usepackage{
booktabs,
datetime,
dtk-logos,
graphicx,
multicol,
pgfplots,
ragged2e,
tabularx,
tikz,
wasysym,
multirow,
float,
caption,
subcaption,
amsmath,
mathptmx,
animate
}

\usepackage[scaled=0.9]{helvet}
\usepackage{courier}

\usefonttheme[onlymath]{serif}

\definecolor{mygreen}{RGB}{113, 166, 70}
\definecolor{myblue}{RGB}{68, 140, 185}
\definecolor{myred}{RGB}{217, 98, 55}
\definecolor{mypurple}{RGB}{83, 65, 126}
\definecolor{solviaveis}{RGB}{188, 207, 241}

\pgfplotsset{compat=1.8}

\usepackage[utf8]{inputenc}
\usepackage[portuguese]{babel}
\usepackage[T1]{fontenc}
\usepackage{newpxtext,newpxmath}
\usepackage{listings}

\lstset{ %
language=[LaTeX]TeX,
basicstyle=\normalsize\ttfamily,
keywordstyle=,
numbers=left,
numberstyle=\tiny\ttfamily,
stepnumber=1,
showspaces=false,
showstringspaces=false,
showtabs=false,
breaklines=true,
frame=tb,
framerule=0.5pt,
tabsize=4,
framexleftmargin=0.5em,
framexrightmargin=0.5em,
xleftmargin=0.5em,
xrightmargin=0.5em
}



%-=-=-=-=-=-=-=-=-=-=-=-=-=-=-=-=-=-=-=-=-=-=-=-=
%        LOADING TIKZ LIBRARIES
%-=-=-=-=-=-=-=-=-=-=-=-=-=-=-=-=-=-=-=-=-=-=-=-=

\usetikzlibrary{
backgrounds,
mindmap
}

%-=-=-=-=-=-=-=-=-=-=-=-=-=-=-=-=-=-=-=-=-=-=-=-=
%        BEAMER OPTIONS
%-=-=-=-=-=-=-=-=-=-=-=-=-=-=-=-=-=-=-=-=-=-=-=-=

\setbeameroption{show notes}

%-=-=-=-=-=-=-=-=-=-=-=-=-=-=-=-=-=-=-=-=-=-=-=-=
%        BEAMER COMMANDS
%-=-=-=-=-=-=-=-=-=-=-=-=-=-=-=-=-=-=-=-=-=-=-=-=


%-=-=-=-=-=-=-=-=-=-=-=-=-=-=-=-=-=-=-=-=-=-=-=-=
%
%	PRESENTATION INFORMATION
%
%-=-=-=-=-=-=-=-=-=-=-=-=-=-=-=-=-=-=-=-=-=-=-=-=

\title{Análise de sensibilidade}
\subtitle{DCE692 - Pesquisa Operacional}
%\date{\small{\jobname}}
\author{\texttt{Iago Carvalho}}
\institute{\texttt{Departamento de Ciência da Computação}}

\hypersetup{
pdfauthor = {Iago A. Carvalho},      
pdfsubject = {Pesquisa Operacional},
pdfkeywords = {},  
pdfmoddate= {D:\pdfdate},          
pdfcreator = {WriteLaTeX}
}

\begin{document}

\begin{frame}
\titlepage

\end{frame}

%% --------------------------------------------------------

\begin{frame}{Análise de sensibilidade}
 
Na análise de sensibilidade, nós estamos interessados em investigar variações da solução ótima de um problema de programação linear

\vspace{0.5cm}

Também estamos interessados em analisar diferentes valores para os coeficientes do modelo
\begin{itemize}
    \item Matriz $A$ e vetores $b$ e $c$
\end{itemize}

\vspace{0.5cm}

Variações nestes coeficientes podem ter 3 diferentes resultados
\begin{enumerate}
    \item A solução ótima não é alterada
    \item A solução ótima atual torna-se inviável
    \item É possível encontrar outra solução com valor melhor que o ótimo atual
\end{enumerate}

\end{frame}

%% --------------------------------------------------------

\begin{frame}{Análise de sensibilidade}

Um dos objetivos da análise de sensibilidade é identificar os parâmetros sensíveis de um modelo
\begin{itemize}
    \item Aqueles cujos valores não podem ser alterados sem alterar a solução ótima
\end{itemize}

\vspace{0.5cm}

Para os coeficientes não sensíveis $c$ da função objetivo, também é interessante identificarmos o intervalo de valores para o qual a solução objetivo permanecerá inalterada
\begin{itemize}
    \item O valor da solução ótima pode mudar, mas os valores das variáveis permanecem estáveis
\end{itemize}
\end{frame}

%% --------------------------------------------------------

\begin{frame}{Preço sombra}

O preço sombra representa é o valor máximo que pode ser pago para dar mais folga a uma restrição
\begin{itemize}
    \item Equivalente a comprar unidades adicionais de algum recurso 
\end{itemize}

\vspace{0.5cm}

O preço sombra é equivalente ao valor ótimo da variável dual associada a uma restrição

\end{frame}

%% --------------------------------------------------------

\begin{frame}{Preço sombra}

Existem 3 maneiras de computar o preço sombra de uma restrição
\begin{enumerate}
    \item Realizando algebrismos com o tableau ótimo do modelo de programação linear
    \item Reotimizar o modelo alterando uma unidade no lado direito da restrição
    \item Resolver o dual e verificar o valor da variável associada a restrição
\end{enumerate}

\end{frame}

\end{document}